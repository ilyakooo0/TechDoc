\section{Введение}

\subsection{Наименование программы}

\subsubsection{Наименование программы на русском языке}

Редактор векторных изображений на платформе iOS <<Dotrix>>.

\subsubsection{Наименование программы на английском языке}

Vector graphics editor for iOS  <<Dotrix>>.

\subsection{Краткая характеристика области применения}

На момент разработки все приложения для создания векторной графики на мобильных устройствах, основным методом ввода на которых является касание экрана пальцами, базируются на точных манипуляциях с элементами, что затруднительно с данной моделью взаимодействия. 

Ни в одном из приложений на рынке не представлена возможность проверить созданную в приложении иллюстрацию в реальном пространстве, что сильно замедляет процесс создания графики, предназначенной для использования в на физических носителях (вывески, логотипы). 

Данное приложение нацелено в первую очередь упростить процесс создания элементов простых графических иллюстрация для различных целей, оптимизируют модель взаимодействия пользователя под мобильные устройства с возможностью просмотра изображения в дополненной реальности.

Также преследуется цель предоставления пользователю возможности удобным способом работать напрямую с кривыми Безье, не абстрагируясь от них.

\subsection{Основание для разработки}

Приказ декана факультета компьютерных наук Национального исследовательского университета «Высшая школа экономики» № 2.3-02/1212-01 от 12.12.2017 <<Об утверждении тем, руководителей курсовых работ студентов образовательной программы Программная инженерия факультета компьютерных наук>>.
