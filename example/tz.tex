% creator: ilyakooo0
% extra credit: dashared
% version: 1.0.1

\documentclass{../TechDoc}

\title{Редактор векторных изображений на платформе iOS <<Dotrix>>}
\author{Студент группы БПИ 174}{И. И. Костюченко}
\academicTeacher{Доцент департамента программной инженерии факультета компьютерных наук, канд. техн. наук}{Х. М. Салех}

\documentTitle{Техническое задание}
\documentCode{RU.17701729.04.03-01 ТЗ 01-1}

\begin{document}
    \maketitle
    
    \tableofcontents
    
    \section{Введение}

\subsection{Наименование программы}

\subsubsection{Наименование программы на русском языке}

Редактор векторных изображений на платформе iOS <<Dotrix>>.

\subsubsection{Наименование программы на английском языке}

Vector graphics editor for iOS  <<Dotrix>>.

\subsection{Краткая характеристика области применения}

На момент разработки все приложения для создания векторной графики на мобильных устройствах, основным методом ввода на которых является касание экрана пальцами, базируются на точных манипуляциях с элементами, что затруднительно с данной моделью взаимодействия. 

Ни в одном из приложений на рынке не представлена возможность проверить созданную в приложении иллюстрацию в реальном пространстве, что сильно замедляет процесс создания графики, предназначенной для использования в на физических носителях (вывески, логотипы). 

Данное приложение нацелено в первую очередь упростить процесс создания элементов простых графических иллюстрация для различных целей, оптимизируют модель взаимодействия пользователя под мобильные устройства с возможностью просмотра изображения в дополненной реальности.

Также преследуется цель предоставления пользователю возможности удобным способом работать напрямую с кривыми Безье, не абстрагируясь от них.

\subsection{Основание для разработки}

Приказ декана факультета компьютерных наук Национального исследовательского университета «Высшая школа экономики» № 2.3-02/1212-01 от 12.12.2017 <<Об утверждении тем, руководителей курсовых работ студентов образовательной программы Программная инженерия факультета компьютерных наук>>.

    
    \section{Назначение разработки}

\subsection{Функциональное назначение}

Функциональным назначением приложения является обеспечение процесса создания, редактирования и сохранения векторных изображений с возможностью просмотра созданного изображения на поверхностях реальных объектов посредством дополненной реальности. Редактирование изображения производится методами, спроектированными с учетом особенностей взаимодействия с мобильными устройствами, позволяющими способом работать напрямую с кривыми Безье, не абстрагируясь от них, сохраняя простую модель взаимодействия с пользователем.

\subsection{Эксплуатационное назначение}

Данное приложение может быть использовано дизайнером или разработчиком для создания собственных элементов графического интерфейса или любым человеком, имеющим необходимость в создании простых иллюстраций. Просмотр изображений в дополненной реальности ориентирован на создание графики для печати, например создание изображения для печати на футболке или логотип, предназначенный для печати на вывеске.

Также, отсутствие абстракции от кривых Безье с постоянной возможностью редактирования точек созданной или создаваемой кривой без их скрытия позволяет пользователю быстрее овладеть интуитивным понимаем принципа работы кривых Безье, что делает приложение пригодным для обучения в сфере векторной графики.

    \section{Требования к программе}
    
    \subsection{Требования к функциональным характеристикам}
    
    Приложение должно реализовывать спроектированную с учетом особенностей использования мобильных устройств системы взаимодействия с пользователем. При необходимости, если этого будет требовать реализация системы взаимодействия, проектирование и реализация собственного формата хранения векторных изображений с последующей реализацией создания, редактирования этого формата и реализация алгоритмов и методов перевода данного формата в распространенные форматы хранения изображения, включая векторные, позволяющие использовать изображения, созданные в данном формате в ряде случаев, разработанных для работы с данными форматами. 

Приложение также должно реализовывать возможность просмотра созданного изображения на поверхностях реальных объектов, посредством дополненной реальности, позволяющую пользователю быстро опробовать иллюстрацию в контексте финального применения, что позволяет ускорить процесс создания этого изображения.

Разрабатываемое приложение должно: 

\begin{enumerate}
    \item создавать и сохранять новые векторные изображения;
    \item редактировать ранее созданные векторные изображения;
    \item изменять параметры изображения:
    \begin{enumerate}
        \item цвет линий;
        \item толщину линий.
    \end{enumerate}
    \item добавление новых элементов в изображение;
    \item переводить изображение в другие форматы;
    \item переводить созданное изображение в различные форматы:
    \begin{enumerate}
        \item png (растровое)
        \item jpeg (растровое)
        \item pdf (векторное)
        \item svg (векторное)
    \end{enumerate}
    \item помещать изображения на плоскости реального пространства
    \begin{enumerate}
        \item распознавать плоскости реального пространства, посредством различных сенсоров мобильных устройств, таких как камера, акселерометр и гироскоп
        \item представлять пользователю возможность помещать созданное в приложении изображение на распознанных плоскостях
        \item позволять пользователю сделать фотографию реального пространства с помещенным в него изображением
    \end{enumerate}
\end{enumerate}

        
    \subsection{Требования к интерфейсу}
        
    Приложение должно иметь оптимальный интерфейс, позволяющий пользователю работать с программой с минимальной предварительной подготовкой. 

Интерфейс должен позволять:
\begin{enumerate}
    \item открывать ранее созданные файлы;
    \item с минимальными усилиями от пользователя вносить в изображение желаемые изменения;
    \item сохранять файлы;
    \item переводить изображение в желаемый формат;
    \item просматривать изображения в на плоскости реального объекта.
\end{enumerate}

 
    \subsection{Требования к формату входных и выходных данных}
    
    \subsubsection{Входные данные}

    Одна из основных задач приложения -- облегчить процесс создания простых иллюстраций учитывая особенности платформы. На устройствах платформы iOS отсутствуют кнопки для работы с приложениями и в большинстве случаев отсутствует возможность точных манипуляций, так как все взаимодействие с приложением пользователь производит пальцами. Apple, производитель платформы, рекомендует использовать ``прямые манипуляции'', когда пользователь взаимодействует напрямую с контентом.
    
    Исходя из этого, приложение должно предоставлять пользователю возможность редактирования изображений, обходя ограничения мобильных устройств, не базируясь на точности выбора точек.

    Приложение должно предоставлять интерфейс, позволяющий пользователю задавать и изменять сегменты векторного изображения путем выбора и манипуляций ключевых точек изображения. Манипуляции производиться должны напрямую путем изменения положения точек.

    \subsubsection{Выходные данные}
    
    Приложение должно позволять пользователю сохранять изображения в различных форматах с различными параметрами.
    
    При необходимости, если этого потребует реализация, приложение должно иметь возможность создания, и редактирования файлов собственного формата.
    
    \subsection{Условия эксплуатации}
    
    \subsubsection{Климатические условия}
    
    Климатические условия сопадают с климатическими условиями эксплуатации устройства.

    \subsubsection{Требования к пользователю}
    
    Пользователь должен иметь базовое представление об основных принципах векторной графики и операционной системы iOS. Также пользователь должен обладать базовыми знаниями работы кривых Безье.

    \subsection{Требования к составу и параметрам технических средств}
    
    Для корректной работы приложения необходимо устройство с процессором A9 или новее, так как у более ранних процессоров не хватает мощности для обработки данных в распознавании плоскостей реальных объектов, что делает использование дополненной реальности не практичным.
    

    \subsection{Требования к информационной и программной совместимости}
    
    На устройстве должна быть установлен операционная система iOS 11.3 или новее, так как в более ранних версия операционной системы не было возможности распознавания вертикальных поверхностей, что используется в данном приложении.
    
    \subsection{Требования к составу сетевых средств}
    
    У устройства должен быть доступ к сети интернет для скачивания и установки данного приложения.

    \subsection{Требования к программной документации}

    В состав программной документации должны входить следующие компоненты:
\begin{enumerate}
    \item Техническое задание (ГОСТ 19.201-78)
    \item Программа и методика испытаний (ГОСТ 19.301-78)
    \item Пояснительная записка (ГОСТ 19.404-79)
    \item Руководство оператора (ГОСТ 19.505-79)
    \item Текст программы (ГОСТ 19.401-78)
\end{enumerate}
    

    \subsection{Требования к маркировке и упаковке}
    
    Приложение должно размещаться в магазине «App Store» в соответствии с авторскими правами и может быть загржуено оттуда потенциальным пользователем.

    \section{Стадии и этапы разработки}
    
    \begin{enumerate}
        \item техническое задание:
        \begin{enumerate}
            \item этапы разработки:
            \begin{enumerate}
                \item обоснование необходимости разработки программы; 
                \item постановка задачи; 
                \item сбор исходных материалов; 
                \item выбор и обоснование критериев эффективности и качества разрабатываемой программы; 
                \item обоснование необходимости проведения научно-исследовательских работ; 
            \end{enumerate}
            \item разработка и утверждение технического задания:
            \begin{enumerate}
                \item определение требований к программе; 
                \item определение стадий, этапов и сроков разработки программы и документации на неё; 
                \item согласование и утверждение технического задания; 
            \end{enumerate}
        \end{enumerate}
    \item технический проект:
    \begin{enumerate}
        \item разработка технического проекта:
        \begin{enumerate}
            \item уточнение структуры входных и выходных данных; 
            \item разработка алгоритма решения задачи; 
            \item определение формы представления входных и выходных данных; 
            \item разработка структуры программы; 
            \item окончательное определение конфигурации технических средств. 
        \end{enumerate}
        \item утверждение технического проекта:
        \begin{enumerate}
            \item разработка пояснительной записки; 
            \item согласование и утверждение технического проекта. 
        \end{enumerate}
    \end{enumerate}
    \item рабочий проект:
    \begin{enumerate}
        \item разработка программы:
        \begin{enumerate}
            \item программирование и отладка программы. 
        \end{enumerate}
        \item разработка программной документации:
        \begin{enumerate}
            \item разработка программных документов в соответствии с требованиями гост 19.101-77. 
        \end{enumerate}
        \item испытания программы:
        \begin{enumerate}
            \item разработка, согласование и утверждение порядка и методики испытаний; 
            \item корректировка программы и программной документации по результатам испытаний.
        \end{enumerate}
    \end{enumerate}
    \end{enumerate}

    \section{Порядок контроля и приемки}
    
    Контроль и приемка разработки осуществляются в соответствии с документом «Программа и методика испытаний».
    
    \section{Технико-экономические показатели}
    
     На сегодняшний момент количество продуктов и цифровых, и не цифровых, требующих какого-либо графического дизайна растет с каждым днем. По данным бюро трудоустройства США к 2026 году количество рабочих мест дизайнеров возрастет на 11100 мест. Вслед за увеличением числа дизайнеров, возрастет и спрос на соответствующие приложения. С увеличением мощности и возможности работы на мобильных устройствах, большее количество людей будет переходить с работы на ПК на работу на мобильных устройствах. Предполагается, что данное приложение будет пользоваться большим спросом среди дизайнеров, выполняющих простые графические иллюстрации, и людей, имеющих желание приобрести интуитивное понимание работы векторной графики и кривых Безье.
        
    \registrationList
        
\end{document}